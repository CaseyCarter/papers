% Define section from the C++ standard that can be indexed
% using its dotted identifer. That is:
%
%  \cxxsec{basic.def.odr}{6.2}
%
% This is used to make references to sections of the C++ Standard
% that are not labeled within this document.
\newcommand{\cxxsec}[2]{%
  \expandafter\def\csname #1 \endcsname{#2}%
}

% Generate a reference to the section with the given id. This
% expands to the full chapter/section/subsection number declared
% by \cxxsec. For example:
%
%  \cxxref{basic.def.odr}
%
% Expands to the string 3.2.
\newcommand{\stdcxxref}[1]{%
  \href{http://eel.is/c++draft/#1}{[#1]}%
}

\newcommand{\cxxref}[1]{%
  \stdcxxref{#1}%
}

% Generate a reference to the section with the given id, enclosed within
% parentheses, before which linebreaking is inhibited.
\newcommand{\cxxiref}[1]{\nolinebreak[3] (\cxxref{#1})}

\cxxsec{basic.compound}{6.8.2}

\cxxsec{customization.point.object}{16.4.2.2.6}

\cxxsec{iterator.synopsis}{23.2}
\cxxsec{iterator.concept.winc}{23.3.4.4}
\cxxsec{iterator.concept.sizedsentinel}{23.3.4.8}
\cxxsec{iterator.concept.bidir}{23.3.4.12}

\cxxsec{ranges}{24}
\cxxsec{ranges.syn}{24.2}
\cxxsec{range.sized}{24.4.3}
\cxxsec{range.subrange}{24.5.3}
